\documentclass[11pt, xcolor={dvipsnames}]{beamer} % English
%\documentclass[11pt, xcolor={dvipsnames}, french]{beamer} % Fran�ais



%%%%%%%%%%
%%%%% import some useful packages
%%%%%%%%%%

%\usepackage[utf8]{inputenc}
\usepackage[latin1]{inputenc}
\usepackage[frenchb]{babel}
%\usepackage[english]{babel}
%\usepackage{frenchle}

%\usepackage{fancybox}
\usepackage{graphicx}
\usepackage{xcolor,colortbl}

\newcommand{\fixme}[1]{\textcolor{red}{\textbf{FIXME}: #1}}

\newcommand{\greyText}[1]{\textcolor{Gray}{#1}}


\usetheme{Madrid}
%\usetheme{Singapore}
%\usetheme{Berlin}
\useoutertheme{infolines}  % add footline with title and slide number



%%%%%%%%%%
%%%%% Document's metadata (title, author, date,...)
%%%%%%%%%%


\title[short title]{Presentation title}
\subtitle[]{Presentation (optional) subtitle}

\author[F. Lastname]{Firstname Lastname}
%\author[F. Lastname]{Firstname Lastname\\~\texttt{firstname.lastname@univ-rennes1.fr}}
%\institute[]{Universit� de Rennes 1, France}
%\date{20 juin 2017}

\setbeamertemplate{navigation symbols}{} % Get rid of navigation symbols




%%%%%%%%%%
%%%%% The real document begins
%%%%%%%%%%

\begin{document}



%%%%%%%%%%
%%%%% First slide with title
%%%%%%%%%%
\begin{frame}[plain]
\maketitle
\end{frame}



%%%%%%%%%%
%%%%% The presentation can be organized into a hierarchy of sections.
%%%%% This may be used in the outline, but otherwise usually not displayed
%%%%%%%%%%
\section{Fist section (not displayed)}



\begin{frame}
 \frametitle{Slide title}
 
Some concent.

\end{frame}


\begin{frame}
 \frametitle{Second slide: blocks}


\begin{block}{Block title}
 Block content
\end{block}

\begin{alertblock}{Alertblock title}
 Block content
\end{alertblock}

\end{frame}



\end{document}
